\chapter{ASM.js}

asm.js je raziskovalni projekt pri Mozilli, ki definira podmnožico JavaScripta. Cilj podmnožice, kot jo definira asm.js, je preprost način optimizacije izvajanja JavaScript kode znotraj brskalnikov. 

C++ prevajalnika Emscripten in Mandreel že znata generirati JavaScript kodo, ki jo defenira asm.js. Patent, ki ga uporabljata Emscripten in Mandreel sta ponazarjanje spomina s samostojno instanco (singleton) tipiziranega spomina in uporaba bitnih operatorjev za spremenljivke, ki se obnašajo kot cela števila (integers) v C++.

Prevajanje v JavaScript ni nič novega. Leta 2006 je podjetje Google izdalo Google Web Toolkit (GWT), ki poleg drugih stvari, lahko  tudi prevaja izvorno kodo iz programskega jezika Java v JavaScript. Od leta 2006 se je pojavilo kar nekaj podobnih prevajalnikov za že obsotječe programske jezike (C++, C\#), kot tudi za nove jezike kot so na primer CoffeeScript, TypeScript in Dart.

Problem projektov kot je GWT je v tem, da ni standardne dokumentacije, ki bi izdelovalcem JavaScript pogonom omogočilo optimizacije. Zato je tudi na primer znano, da GWT aplikacije v brskalniku Google Chrome tečejo malce hitreje kot v drugih brskalnikih. Razlog je v tem, da sta tako GWT in Chrome razvita pod isto streho (Google) in je veliko več interne komunikacije, ki pa je ostali brskalniki niso deležni. asm.js dokumentira vse možne pohitritve in navodila za pohitritve da na voljo vsem brskalnikom.

asm.js se izogiba potencialnih upočasnitev v kodi, saj nima spremenljivk z mešanimi tipi. Ime knjižnice izvira v dejstvu, da asm.js izvorna koda izvaja zgolj nizko nivojske izračune, ki so podobni tistim, ki jih izvajajo zbirniki. To pa je točno to, kar preveden C/C++ potrebuje.

Optimizacije v času izvajanja:

1. Tipi spremenljivk se pokažejo med preverjanjem tipov. To omogoča prevajanje v naprej (ahead of time) in ne samo ob pravem času (just in time).

2. JavaScript pogon ima garancijo, da se tipi spremenljivk med izvajanjem ne bodo spreminjali. S tem lahko pogon generira bolj preprosto in bolj učinkovito kodo.

3. Sistem tipov v asm.js olajša globalno strukturiranje programa (klici funkcij, dostop do spomina)

Izvorna koda, ki uporablja asm.js je še vedno dvakrat počasnejša od materne (native) kode napisane v nižje nivojskih jezikih kot je C, vendar se bo s časoma asm.js še dodatno pohitrila.

Ker je asm.js koda pod množica JavaScripta lahko že danes teče v vseh brskalnikih.  

Kaj trenutno še ni podprto? C++ izjeme, setjmp/longjmp.

Popolnoma podpra sta trenutno samo C in C++, drugi jeziki so poprti le deloma in niso deležni enakih pohitritev in optimizaij.

Dinamični jeziki, kot so Python, Ruby in Lua, so še v zgodnjih stadijih razvoja in potrebujejo še veliko dela preden bodo uporabni.

Dodaten problem pri jezkih, kot sta Java in C\# je v tem, da se veliko optimizacij naredi navidezni stroj na nivoju byte kode. Te optimizacije se izgubijo, če prevajalnik prevaja iz izvorne kode v izvorno kodo.

Edin način kako dobiti boljše pohitritve v teh jezikih je prevajanje celotnih navideznih strojev... To zgleda kot edini način kako izvajati večino jezikov s perfektno semantiko in maksimalno hitrostjo.

% http://kripken.github.io/mloc_emscripten_talk/\#/40
% http://mozakai.blogspot.com/2013/06/what-asmjs-is-and-what-asmjs-isnt.html
