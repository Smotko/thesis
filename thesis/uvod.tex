\chapter{Uvod}

Še pred nekaj leti je bilo poganjanje grafično intenzivnih aplikacij domena dragih delovnih postaj. Danes je tega zmožen vsak osebni računalnik in tudi naprave, ki jih prenašamo v žepih.

Zmogljive mobilne naprave postajajo vse bolj vsakdanje. Pametni telefoni imajo v sebi več procesorske moči, kot namizni računalniki izpred parih let. Poleg procesorske moči praviloma vsebujejo tudi grafične procesne enote, ki jih razvijalci lahko izkoristijo za razvoj grafično intenzivnih aplikacij. Poleg telefonov pa so se pojavili tudi tablični računalniki, ki imajo praviloma še boljše karakteristike kot pametni telefoni. 

Med posameznimi proizvajalci telefonov in tablic obstajajo velike razlike v razvojnem okolju. Vsak izmed mobilnih operacijskih sistemov uporablja drug programski jezik za razvoj domorodnih aplikacij pa tudi pri knjižnicah se pojavljajo razlike (npr. OpenGL ES, Direct3D). Razvoj grafične aplikacije, ki bi jo napisali enkrat in bi delovala povsod, je tako skorajda nemogoč.

Problem postane še težji, če želimo poleg vseh mobilnih naprav podpreti še namizne računalnike. Sedaj imamo poleg različnih programskih jezikov in knjižnic, še različne možnosti interakcije z uporabnikom - vnos z dotikom na mobilnih napravah in vnos z miško in tipkovnico na namiznih računalnikih.

\section{Potek diplomske naloge}

Cilj diplomske naloge je pregledati in preizkusiti možnosti za premostitev razlik med različnimi platformami. Na začetku si bomo ogledali različne platforme, njihove razlike in skupne točke. Osredotočili se bomo na mobilne platforme, vendar se bomo dotaknili tudi namiznih računalnikov in računalnikov na eni sami ploščici.

V nadaljevanju diplomske naloge si bomo ogledali nekatera odprtokodna in plačljiva orodja za premostitev razlik med platformami. Preučili bomo spletne tehnologije in orodja, ki temeljijo na prevodih med različnimi programskimi jeziki. Zanimale nas bodo metode, ki temeljijo na uporabi jezika C++, ter orodja, ki ponujajo celotno razvojno okolje. Ogledali si bomo tudi orodje, ki uporablja namenski programski jezik.

Ena izmed glavnih ovir pri razvoju grafično intenzivnih aplikacij s pomočjo spletnih tehnologij je hitrost izvajanja programa znotraj spletnega brskalnika. Zato bomo raziskali načine za pospešitev JavaScript izvorne kode.

Grafično intenzivne aplikacije za tekoče delovanje uporabljajo vedno višje stopnje paralelnosti. S tem razlogom si bomo ogledali procesiranje na grafičnih procesnih enotah.

V zadnjem delu diplomske naloge bomo opisali nekaj primerov grafično intenzivnih aplikacij, ki so bile razvite z namenom pokazati prednosti in slabosti uporabljenih orodij. Razložili bomo uporabljene metode, njihove prednosti in slabosti.
