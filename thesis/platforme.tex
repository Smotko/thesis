\chapter{Pregled platform}

Danes lahko grafično intenzivne aplikacije poganjamo na ogromnem naboru različnih naprav. Grafično pospeševanje izrisovanja 3D objektov, ki je bilo še pred kratkim omejeno na namizne računalnike in konzole, je sedaj možno tudi na mobilnih telefonih in grafičnih tablicah. Razvoj aplikacij za mobilne platforme je nekoliko bolj zahteven, saj moramo poleg okrnjenega nabora ukazov in slabše strojne zmogljivosti, paziti tudi na porabo baterije.

\subsection{Namizni računalniki}

\subsection{Mobilne platforme} 

Na trgu najdemo pester nabor mobilnih platform. Največji igralci v času pisanja so Google s svojim odprtokodnim sistemom Android, Apple s svojim sistemom iOS, nekaj tržnega deleža pa imata tudi podjetji Microsoft, z mobilno različico opercijskega sistema Windows (Windows 7, 8 RT, Phone), ter podjetje BlackBerry z istoimenskim naborom pametnih telefonov namenjenim predvsem poslovnim uporabnikom.

Poleg že obstoječih pa bodo v bližnji prihodnosti na trg stopili tudi novi igralci. Fundacija Mozilla je razvila Firefox OS, podjetje Canonical pa pripravlja različico Ubuntu operacijskega sistema, ki bo delovala tudi na mobilnih telefonih in tablicah.

V zadnjem letu so postale zanimivi računalniki s čipom na eni sami ploščici (prevedi single-board computer). Tak primer je Rasperry Pi, računalnik, ki je bil razvit s namenom promoviranja učenja osnovnih temeljev računalništva.

\subsection{iOS}

Prvi iPhone je bil predstavljen 9. januarja 2007. Od ostalih mobilnih naprav na tržišču se je razlikoval z dodelanim uporabniškim vmesnikom in zaslonom občutljivim za večprstne dotike. Bil je tudi eden izmed prvih mobilnih telefonov brez tipkovnice - uporabnik je do vseh funkcij telefona dostopal preko na dotik občutljivega zaslona.

V nasljednih letih je podjetje Apple Inc. dodalo prvemu telefonu nekaj dodatnih funkcij kot so 3g povezljivost, izboljšana zadnja kamera, zaslon z višjo resolucijo, dvo jederni procesor itd.

Naprave danes podpirajo OpenGL ES 1.1 in 2.0.

Razvoj aplikacij poteka v jeziku Objective C (objektni C), vendar je možen samo na sistemih z Mac OSX operacijskim sistemom.

\subsection{Android}

Android je operacijski sistem, ki temelji na Linux jedru. Operacijski sistem je razvilo podjetje Android Inc., s finančno pomočjo Googla, ki je leta 2005 primarno podjetje tudi kupil. Prvi mobilni telefon z Android operacijskim sistemom je bil prodan oktobra 2008.

Izvorna koda operacijskega sistema je odprta in dostopna pod Apache licenco.

Android podpira OpenGL ES 1.1 in 2.0 od verzije 2.2 dalje.
\subsection{Windows}
\subsection{Firefox OS}
\subsection{RaspberryPi}
\subsection{Ubuntu}

\section{Zmogljivosti}

Edina skupna točka vseh omenjenih 

OpenGL zmonžnosti itd.




